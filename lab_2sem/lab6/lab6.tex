\documentclass[a4paper,12pt]{report}

\usepackage{cmap}
\usepackage[T2A]{fontenc}
\usepackage[utf8]{inputenc}
\usepackage[russian]{babel}
\usepackage{amsmath,amsfonts,amssymb}
\usepackage{graphicx}
\usepackage{sidecap}
\usepackage{wrapfig}


\begin{document} 

\begin{titlepage} 

\begin{center} 

\large Федеральное государственное автономное образовательное учреждение высшего образования «Санкт-Петербургский государственный электротехнический университет «ЛЭТИ» им. В.И. Ульянова (Ленина)»
	
кафедра физики\\[5cm] 


\huge ОТЧЕТ\\ по лабораторной работе № 16\\[0.5cm] 
\large <<ИЗМЕРЕНИЕ МАГНИТНОГО ПОЛЯ ЗЕМЛИ>>\\[3.7cm]

\begin{minipage}{1\textwidth} 
    \begin{flushleft} 
        \emph{Автор:} Стукен В.А.\\
        \emph{Группа:} 2307\\
        \emph{Факультет:} ФКТИ\\
        \emph{Преподаватель:} Чурганова С.С. 
    \end{flushleft} 
\end{minipage} 

\vfill 

Санкт-Петербург, 2023\\
{\large \LaTeX} 

\end{center} 

\thispagestyle{empty} 
\end{titlepage} 


\section*{Протокол измерений}

\begin{flushleft}
    
    \resizebox{8cm}{!}{ 
        \begin{tabular}{|l|l|l|}
                \hline
                 №&$U_v$ & $U_g$ \\
                \hline
                1 & &\\
                \hline
                2& &\\
                \hline
                3& &\\
                \hline
                4& &\\
                \hline
                5& &\\
                \hline
                6& &\\
                \hline
                7& &\\
                \hline
                8& &\\
                \hline
                9& &\\
                \hline
                10& &\\
                \hline
        \end{tabular}
    }
\end{flushleft}
\newpage
\section*{Ответы на вопросы}

\begin{itemize}
    \item Вопрос №9: \textbf{Производит ли сила Ампера работу? Ответ обоснуйте.}\\
    Да, конечно, производит. При малом перемещении $dr$ элемента $dl$ проводника с током $I$ работа силы Ампера $dF$ равна:
    \[\delta A = d\vec{F}d\vec{r} \]
    Или:
    \[ \delta A = Id\Phi_m \]
    При малом перемещении работа Силы Ампера равна:
    \[ \delta A = \int_l Id\Phi_m \]
    Если в проводнике $I = const$ и он совершает конечное перемещение, то 
    \[ \delta A = I\Phi_m \]
    ,где $\Phi_m$ - магнитный поток сквозь поверхность, прочерчиваемую всем проводником при малом перемещении этого проводника.
    \item Вопрос №59: \textbf{Какие вещества называют ферромагнетиками? Как они ведут себя во внешнем магнитном
    поле?}
    \par
    \textit{Ферромагнетики} – это вещества, в которых магнитные моменты атомов или ионов находятся в состоянии самопроизвольного магнитного упорядочения, 
    причем результирующие магнитные моменты каждого из доменов отличны от нуля. 
    При воздействии внешнего магнитного поля магнитные моменты доменов приобретают преимущественное ориентирование в направлении этого поля и 
    ферромагнитное вещество намагничивается. Ферромагнитные вещества характеризуются большим значением магнитной восприимчивости (>> 1), 
    а также ее нелинейной зависимостью от напряженности магнитного поля и температуры, способностью намагничиваться до насыщения при обычных температурах даже в слабых магнитных полях, гистерезисом — зависимостью магнитных свойств от предшествующего магнитного состояния, точкой Кюри, т. е. температурой, выше которой материал теряет ферромагнитные свойства. К ферромагнитным веществам относятся железо, никель, кобальт, их соединения и сплавы, а также некоторые сплавы марганца, серебра, алюминия. Ферромагнитные свойства у вещества могут возникать лишь при достаточно большом значении обменного взаимодействия, что характерно для кристаллов железа, кобальта, никеля и др.
\end{itemize}


\end{document}