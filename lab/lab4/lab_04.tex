\documentclass[a4paper,12pt]{report}

\usepackage{cmap}
\usepackage[T2A]{fontenc}
\usepackage[utf8]{inputenc}
\usepackage[russian]{babel}
\usepackage{amsmath,amsfonts,amssymb}
\usepackage{graphicx}
\usepackage{sidecap}
\usepackage{wrapfig}
% \usepackage{pgfplots} 
% \pgfplotsset{compat = 1.18,width = 15cm}

\begin{document} 

\begin{titlepage} 

\begin{center} 

\large Федеральное государственное автономное образовательное учреждение высшего образования «Санкт-Петербургский государственный электротехнический университет «ЛЭТИ» им. В.И. Ульянова (Ленина)»
	
кафедра физики\\[5cm] 


\huge ОТЧЕТ\\ по лабораторной работе № 9\\[0.5cm] 
\large <<ИССЛЕДОВАНИЕ РАЗВЕТВЛЕННЫХ
ЦЕПЕЙ С ПРИМЕНЕНИЕМ КОМПЕНСАЦИОННОГО МЕТОДА
ИЗМЕРЕНИЙ 
>>\\[3.7cm]

\begin{minipage}{1\textwidth} 
    \begin{flushleft} 
        \emph{Автор:} Стукен В.А.\\
        \emph{Группа:} 2307\\
        \emph{Факультет:} ФКТИ\\
        \emph{Преподаватель:} Харитонский П.В. 
    \end{flushleft} 
\end{minipage} 

\vfill 

Санкт-Петербург, 2022\\
{\large \LaTeX} 

\end{center} 

\thispagestyle{empty} 
\end{titlepage} 


\newpage
\section*{Ответы на вопросы}

\begin{itemize}
    \item \textbf{Вопрос №24:}
    	\textit{"Что такое шунт? Для чего его применяют?"} \\
        Шунт - это дополнительное сопротивление, которое подключается последовательно вольтметру или параллельно амперметру, чтобы изменить величину тока, протекающего через измерительный прибор.
    \item \textbf{Вопрос №36:}
        \textit{"Сформулируйте первое правило Кирхгофа. Сколько независимых уравнений можно написать, используя это правило?"}
        Алгебраическая сумма токов ветвей , сходящихся в каждом узле любой цепи равна 0, т.е
         $\sum_{i = 1}^{N} I_i = 0$.
        \\ Соответственно, можно написать столько независимых уравнений для цепи, сколько в ней узлов.
        
\end{itemize}

\newpage

\section*{Протокол измерений}

\begin{flushleft}
    
\resizebox{15cm}{!}{ 
    \begin{tabular}{|l|l|l|l|l|l|l|}
        \hline
           & & 1 & 2 & 3 & 4 & 5 \\
        \hline
           $G_2$    &  $n_0$   &          &         &     &        &        \\ 
        \hline
            $G_1$&   $n_x$  &          &         &     &        &        \\ 
        \hline
            $G_1,R$ &  $n_x$   &          &         &     &        &      \\  
        \hline
    \end{tabular}
}
\end{flushleft}

$E_0$ = 1.275 V; $R_2$ = 3 кОм $n_{max}$ = 10


\newpage
\section*{Обработка результатов измерений}

\subsection*{Рассчитаем среднее значение $n_0$}

\[ \bar{n_0} = \frac{\sum n_{0i}}{N} = 1.22 \]
\[S_{n_0}=\sqrt{\frac{(\bar{n_0}-{n_0}_1)^2+(\bar{n_0}-{n_0}_2)^2+(\bar{n_0}-{n_0}_3)^2+(\bar{n_0}-{n_0}_4)^2+(\bar{n_0}-{n_0}_5)^2}{N-1}} = 0.016\]
\[\bar{S_{n_0}}=\frac{S_{n_0}}{\sqrt{N}}=\frac{0.016}{\sqrt{5}}=0.007\]
\[ \Delta{n_0}_{(\bar{S_{n_0}})}=\bar{S_{n0}}\cdot t_{PN}=0.007\cdot 2.8=0.02 \]
\[ \Delta{\bar{n_0}}=\sqrt{\Delta n_0^2 + \theta^2}=\sqrt{0.02^2 + 0.0005^2}=0.02 \]
\[{n_0}=\bar{n_0}\pm \Delta{\bar{n_0}}=1.22\pm 0.02\]

\subsection*{Рассчитаем среднее значение $n_x$}

\[ \bar{n_x} = \frac{\sum n_{xi}}{N} = 2.316 \]
\[S_{n_x}=\sqrt{\frac{(\bar{n_x}-{n_x}_1)^2+(\bar{n_x}-{n_x}_2)^2+(\bar{n_x}-{n_x}_3)^2+(\bar{n_x}-{n_x}_4)^2+(\bar{n_x}-{n_x}_5)^2}{N-1}} = 0.009\]
\[\bar{S_{n_x}}=\frac{S_{n_x}}{\sqrt{N}}=\frac{0.009}{\sqrt{5}}=0.004\]
\[ \Delta{n_x}_{(\bar{S_{n_x}})}=\bar{S_{nx}}\cdot t_{PN}=0.004\cdot 2.8 = 0.011 \]
\[ \Delta{\bar{n_x}}=\sqrt{\Delta n_x^2 + \theta^2}=\sqrt{0.011^2 + 0.0005^2}=0.011 \]
\[{n_x}=\bar{n_x}\pm \Delta{\bar{n_x}}=2.316\pm 0.011\]

\subsection*{Рассчитаем среднее значение $n_{x'}$}

\[ \bar{n_{x'}} = \frac{\sum n_{x'}}{N} = 2.328 \]
\[S_{n_{x'}}=\sqrt{\frac{(\bar{n_{x'}}-{n_{x'}}_1)^2+(\bar{n_{x'}}-{n_{x'}}_2)^2+(\bar{n_{x'}}-{n_{x'}}_3)^2+(\bar{n_{x'}}-{n_{x'}}_4)^2+(\bar{n_{x'}}-{n_{x'}}_5)^2}{N-1}} = 0.008\]
\[\bar{S_{n_{x'}}}=\frac{S_{n_{x'}}}{\sqrt{N}}=\frac{0.008}{\sqrt{5}}=0.004\]
\[ \Delta{n_{x'}}_{(\bar{S_{n_{x'}}})}=\bar{S_{n_{x'}}}\cdot t_{PN}=0.004\cdot 2.8=0.011 \]
\[ \Delta{\bar{n_{x'}}}=\sqrt{\Delta n_{x'}^2 + \theta^2}=\sqrt{0.011^2 + 0.0005^2}=0.011 \]
\[{n_{x'}}=\bar{n_{x'}}\pm \Delta{\bar{n_{x'}}}=2.328\pm 0.011\]

\subsection*{Рассчитаем среднее значение $\varepsilon_x$}

\[ \bar{\varepsilon_x} = \bar{\varepsilon_0} \frac{\bar{n_x}}{\bar{n_0}} = 2.42 \, V\]
\[ a_{n_0} = \frac{\partial \varepsilon_x}{\partial n_0} = -\frac{\varepsilon_0 \cdot n_x}{n_0^2}\]
\[ a_{n_x} = \frac{\partial \varepsilon_x}{\partial n_x} = \frac{\varepsilon_0}{n_0}\]
\[ \bar{a_{n_0}} = -1.98 \]
\[ \bar{a_{n_x}} = 1.04 \]

\[ \bar{\delta \varepsilon_x} = \sqrt{(\bar{a_{n_0}} \cdot \Delta{\bar{n_0}})^2+(\bar{a_{n_x}} \cdot \Delta{\bar{n_x}})^2} = 0.04\]
\[ \varepsilon_x=\bar{\varepsilon_x}\pm \Delta{\bar{\varepsilon_x}}= 2.42\pm 0.04 \, V \]

\subsection*{Рассчитаем среднее значение $\varepsilon_{x'}$}

\[ \bar{\varepsilon_{x'}} = \bar{\varepsilon_0} \frac{\bar{n_{x'}}}{\bar{n_0}} = 2.43 \, V\]
\[ a_{n_0} = \frac{\partial \varepsilon_{x'}}{\partial n_0} = -\frac{\varepsilon_0 \cdot n_{x'}}{n_0^2}\]
\[ a_{n_{x'}} = \frac{\partial \varepsilon_{x'}}{\partial n_{x'}} = \frac{\varepsilon_0}{n_0}\]
\[ \bar{a_{n_0}} = -2 \]
\[ \bar{a_{n_{x'}}} = 1.05 \]

\[ \bar{\delta \varepsilon_{x'}} = \sqrt{(\bar{a_{n_0}} \cdot \Delta{\bar{n_0}})^2+(\bar{a_{n_{x'}}} \cdot \Delta{\bar{n_x}})^2} = 0.04\]
\[ \varepsilon_{x'}=\bar{\varepsilon_{x'}}\pm \Delta{\bar{\varepsilon_{x'}}}= 2.43\pm 0.04 \, V \]
 
\subsection*{Рассчитаем значение $\varepsilon_{max}$}

\[ \varepsilon_{max} = \varepsilon_0 \frac{n_{max}}{n_0} = 10.45 \, V \]

\subsection*{Найдем внутреннее сопротивление микроамперметра $r_0$, полагая, что $r_1 = r_3 = 0$}

\begin{equation*}
    \begin{cases}
      I_2 - I_3 - I_1 = 0, 
      \\
      I_1(r_1 + r_0) + I_2R_x = \varepsilon_x,
      \\
      I_3r_3 + I_3(R_2-R_x) + I_2R_x = \varepsilon_3.
    \end{cases}
   \end{equation*}

\[ r_0 = \frac{\varepsilon_x - I_2R_x}{I_1} \]
\[r_0 = \frac{\varepsilon_x - \varepsilon_3 + I_3(R_2-R_x)}{I_1}\]
\[ \frac{R_x}{R_2} = \frac{n'}{n_{max}} \]
\[ R_x = 2660 \, \Omega  \]
\[ r_0 = 15\cdot 10^{3} \Omega \]

\section*{Вывод}

В ходе лабораторной работы ознакомились с компенсационным методом измерения на примере ЭДС, с помощтю которого вычислили значение ЭДС источника тока, а также применили правила Кирхгофа для расчета разветвленных цепей.

\end{document}