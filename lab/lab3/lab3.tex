\documentclass[a4paper,12pt]{report}

\usepackage{cmap}
\usepackage[T2A]{fontenc}
\usepackage[utf8]{inputenc}
\usepackage[russian]{babel}
\usepackage{amsmath,amsfonts,amssymb}
\usepackage{graphicx}
\usepackage{sidecap}
\usepackage{wrapfig}

\begin{document} 

\begin{titlepage} 

\begin{center} 

\large Федеральное государственное автономное образовательное учреждение высшего образования «Санкт-Петербургский государственный электротехнический университет «ЛЭТИ» им. В.И. Ульянова (Ленина)»
	
кафедра физики\\[5cm] 


\huge ОТЧЕТ\\ по лабораторной работе № 3\\[0.5cm] 
\large <<Исследование динамики колебательного и вращательного движения>>\\[3.7cm]

\begin{minipage}{1\textwidth} 
    \begin{flushleft} 
        \emph{Автор:} Стукен В.А.\\
        \emph{Группа:} 2307\\
        \emph{Факультет:} ФКТИ\\
        \emph{Преподаватель:} Харитонский П.В. 
    \end{flushleft} 
\end{minipage} 

\vfill 

Санкт-Петербург, 2022\\
{\large \LaTeX} 

\end{center} 

\thispagestyle{empty} 
\end{titlepage} 

\section*{Работа №3 "Исследование динамики колебательного и вращательного движения"}

\begin{wrapfigure}{r}{0.2\textwidth}
    \includegraphics[width=0.2\textwidth]{ustanovka.jpg}
    \label{ris:image}
\end{wrapfigure}

\par
\textit{Цель работы:} Исследование динамики колебательного движения на примере крутильного маятника, определение момента инерции маятника, модуля сдвига материала его подвеса и характеристик колебательной системы с затуханием (логарифмического декремента затухания и добротности колебательной системы).
\textit{Приборы и принадлежности: } физический маятник;
секундомер; масштабная линейка, чертежный треугольник.

Применяемый в работе крутильный маятник представляет собой диск 1, закреплен- ный на упругой стальной проволоке 2, свободный конец которой зажат в неподвижном кронштейне 3. На кронштейне расположено кольцо 4, масса ко- торого известна. Кольцо 4 можно положить сверху на диск 1, изменив тем самым момент инерции ма- ятника. Для отсчета значений угла 
поворота маятника служит градуированная шкала 5, помещенная на панели прибора снизу от диска 1.

\section*{Исследуемые закономерности}

\par

\subsection*{Момент инерции крутильного маятника}
\par
Момент инерции (аналог инертной массы тела при его поступательном движении) - 
физическая величина, характеризующая инертные свойства твердого тела при его вращении. 
\par
Если твердое тело вращается вокруг неподвижной оси, то момент инерции относительно этой оси вычисляется как сумма произведений элементарных масс $\Delta m_i$, составляющих тело, на квадраты их расстояний $r_i$ до оси вращения. В случае сплошного тела сумма в определении момента инерции переходит в интеграл .
\par
Крутильный маятник совершает вращательное колебательное движение вокруг оси, совпадающей с направлением стальной проволоки. Используя основное уравнение динамики вращательного движения, можно определить момент инерции маятника, а также физические величины, описывающием вращательное движение. 

\newpage
\subsection*{Уравнение движения крутильного маятника}

\begin{wrapfigure}{r}{0.4\textwidth}
    \includegraphics[width=0.3\textwidth]{graph.jpg}
    \label{ris:image}
\end{wrapfigure}

При повороте тела, закрепленного на упругом подвесе, в результате деформации сдвига возникает вращающий момент упругих сил. 
Трение в подвесе создает тормозящий момент, пропорциональный скорости движения маятника.
Исследуемый в работе крутильный маятник представляет собой сложную систему (диск с различными креплениями, прикрепленный к проволочному подвесу) с неизвестным моментом инерции $I_d$, 
который представляет собой постоянную часть исследуемой системы. Если на диск маятника положить тело с известным моментом инерции - кольцо с моментом инерции $I_k$ , то момент инерции маятника станет равным $I_d+I_k$ . 

\subsection*{Крутильный маятник как диссипативная система}

Полная энергия колебаний маятника убывает со временем. 
Убывание энергии происходит за счет совершения работы против сил трения. 
Энергия при этом превращается в тепло. 
Помимо коэффициента затухания $\beta$ (или времени затухания $\tau$) и мощности потерь $P_d$ колебательная диссипативная система характеризуется также добротностью $Q$, 
позволяющей судить о способности системы сохранять энергию. 
Добротность численно равна числу колебаний за время $t = \tau\pi$. 
За это время амплитуда колебаний уменьшается в $e^{\pi} \approx 23$ раза, 
а энергия колебаний в $e^{2\pi} \approx 535$ раз, иными словами за это время колебания практически затухают. 
Часто также используется параметр $N_e = \frac{\tau}{T}$ - число колебаний, за которое амплитуда колебаний уменьшается в $e$ раз. 

\newpage
\subsection*{Определение модуля сдвига}

\begin{wrapfigure}{r}{0.3\textwidth}
    \includegraphics[width=0.3\textwidth]{sdvig.jpg}
    \label{ris:image2}
\end{wrapfigure}

Методом крутильных колебаний пользуются для косвенного измерения модуля сдвига $G$ материала подвеса. 
Модуль сдвига характеризует упругие свойства материала и в случае малых деформаций равен силе, действующей на единицу площади $S$ при единичном угле сдвига $\gamma$ касательно сдвигу слоев вещества в месте определения модуля $G$.
Модуль сдвига $G$ связан с модулем Юнга, характеризующим сопротивление материала сжатию или растяжению. Коэффициент Пуассона - отношение поперечное и продольной относительной деформации образца материала и для металлов близок к 0.3. 

\newpage

\section*{Протокол измерений}

\begin{flushleft}
    
\resizebox{16cm}{!}{ 
    \begin{tabular}{|l|l|l|l|l|l|l|l|}
        \hline
           $l$ & $d$ & $D_{ex}$ & $D_{in}$ & $D_0$ & $h_0$ & $m$ & $\rho$\\
        \hline
               &     &          &         &     &        &       & \\ 
        \hline
    \end{tabular}
}
\end{flushleft}

\resizebox{10cm}{!}{
    \begin{tabular}{|l|l|l|l|l|}
        \hline
             № & t_d & t_{0d} & $t_k$ &  $t_{0k}$ \\
        \hline
        1  &   &   &   &   \\ 
        \hline
        2  &   &   &   &   \\
        \hline
        3  &   &   &   &   \\
        \hline
        4  &   &   &   &   \\
        \hline
        5  &   &   &   &   \\
        \hline
    \end{tabular}
}


\newpage
\section*{Ответы на вопросы}

\begin{itemize}
    \item \textbf{Вопрос №9:}
    	\textit{"Физический смысл коэффициента затухания $\beta$?"} \\
        Коэффициент затухания $\beta$ характеризует скорость затухания колебаний. $\beta = \frac{1}{\tau}$
    \item \textbf{Вопрос №40:}
        \textit{"Выведите формулу:"}
         \[ I_d = \frac{I_k}{(\frac{T_{dk}}{T_d})^2-1} \]
         \[ \omega_0 = \frac{2\pi}{T} = \sqrt{\frac{D}{I_{d}}} \]
         Отсюда:
         \[ T = 2\pi\sqrt{\frac{D}{I_d}} \]

\end{itemize}
%! Дописать ответы на вопросы


\newpage

\section*{Обработка результатов измерений}

\subsection*{Найдем $t_d = \bar{t_d} \pm \Delta \bar{t_d}$}

$N=5,\ \ U_{PN}=0.64,\ \ V_{PN}=1.67,\ \ \beta_{PN}=0.51,\ \ t_{PN}=2.8$\\
Промахов нет.\\
\[\bar{t}= \frac{15.66+15.66+15.72+15.76+15.91}{5} =15.742 s\] \\
\[S_{t}=\sqrt{\frac{(\bar{t}-{t}_1)^2+(\bar{t}-{t}_2)^2+(\bar{t}-{t}_3)^2+(\bar{t}-{t}_4)^2+(\bar{t}-{t}_5)^2}{N-1}} = 0.103\]\\

\[\bar{S_{t}}=\frac{S_{t}}{\sqrt{t}}=\frac{0.103}{\sqrt{5}}=0.046\]

\[\Delta{t}_{(R)}=R\cdot B_{PN}=0.25\cdot 0.51=0.128 \, s\]
\[\Delta{t}_{(\bar{S_{t}})}=\bar{S_t}\cdot t_{PN}=0.046\cdot 2.8=0.129 \, s\]
\[\Delta{\bar{t}}=\sqrt{\Delta t^2 + \theta^2}=\sqrt{0.129^2 + 0.01^2}=0.129 \, s\]

\[{t_d}=\bar{t_d}\pm \Delta{\bar{t_d}}=15.74\pm 0.12 \, s\]

\subsection*{Найдем $t_{0d} = \bar{t_{0d}} \pm \Delta \bar{t_{0d}}$}

$N=5,\ \ U_{PN}=0.64,\ \ V_{PN}=1.67,\ \ \beta_{PN}=0.51,\ \ t_{PN}=2.8$\\
Промахов нет.\\

\[\bar{t}= \frac{18.9+18.93+19.06+20.34+20.4}{5} = 19.526 \, s\]

\[S_{t}=\sqrt{\frac{(\bar{t}-{t}_1)^2+(\bar{t}-{t}_2)^2+(\bar{t}-{t}_3)^2+(\bar{t}-{t}_4)^2+(\bar{t}-{t}_5)^2}{N-1}} = 0.773\]

\[ \bar{S_{t}}=\frac{S_{t}}{\sqrt{N}}=\frac{0.773}{\sqrt{5}}=0.346 \]

\[ \Delta{t}_{(R)}=R\cdot B_{PN}=1.5\cdot 0.51=0.765 \]
\[ \Delta{t}_{(\bar{S_{t}})}=\bar{S_t}\cdot t_{PN}=0.346\cdot 2.8=0.969 \]

\[ \Delta{\bar{t}}=\sqrt{\Delta t^2 + \theta^2}=\sqrt{0.969^2 + 0.01^2}=0.969 \, s\]

\[ {t_{0d}}=\bar{t_{0d}}\pm \Delta{\bar{t_{0d}}}=19.5\pm 0.9 \, s\]


\subsection*{Найдем $t_{k} = \bar{t_{k}} \pm \Delta \bar{t_{k}}$}

$N=5,\ \ U_{PN}=0.64,\ \ V_{PN}=1.67,\ \ \beta_{PN}=0.51,\ \ t_{PN}=2.8$\\
Промахов нет.\\

\[\bar{t}= \frac{21.71+21.81+21.84+22+22.13}{5} =21.898 \]
\[S_{t}=\sqrt{\frac{(\bar{t}-{t}_1)^2+(\bar{t}-{t}_2)^2+(\bar{t}-{t}_3)^2+(\bar{t}-{t}_4)^2+(\bar{t}-{t}_5)^2}{N-1}} = 0.166\]

\[ \bar{S_{t}}=\frac{S_{t}}{\sqrt{N}}=\frac{0.166}{\sqrt{5}}=0.074 \]

\[ \Delta{t}_{(R)}=R\cdot B_{PN}=0.42\cdot 0.51=0.214 \]
\[ \Delta{t}_{(\bar{S_{t}})}=\bar{S_t}\cdot t_{PN}=0.074\cdot 2.8=0.207 \]
\[ \Delta{\bar{t}}=\sqrt{\Delta t^2 + \theta^2}=\sqrt{0.207^2 + 0.01^2}=0.207 \]

\[ {t_k}=\bar{t_k}\pm \Delta{\bar{t_k}}=21.9\pm 0.2 \, s\]

\subsection*{Найдем $t_{0k} = \bar{t_{0k}} \pm \Delta \bar{t_{0k}}$}

$N=5,\ \ U_{PN}=0.64,\ \ V_{PN}=1.67,\ \ \beta_{PN}=0.51,\ \ t_{PN}=2.8$\\
Промахов нет.\\

\[\bar{t}= \frac{17.65+17.66+17.69+17.79+17.82}{5} =17.722 \, s\]

\[S_{t}=\sqrt{\frac{(\bar{t}-{t}_1)^2+(\bar{t}-{t}_2)^2+(\bar{t}-{t}_3)^2+(\bar{t}-{t}_4)^2+(\bar{t}-{t}_5)^2}{N-1}} = 0,078\]

\[ \bar{S_{t}}=\frac{S_{t}}{\sqrt{N}}=\frac{0.078}{\sqrt{5}}=0.035 \]

\[ \Delta{t}_{(R)}=R\cdot B_{PN}=0.17\cdot 0.51=0.087 \]

\[\Delta{t}_{(\bar{S_{t}})}=\bar{S_t}\cdot t_{PN}=0.035\cdot 2.8=0.098\]

\[ \Delta{\bar{t}}=\sqrt{\Delta t^2 + \theta^2}=\sqrt{0.098^2 + 0.01^2}=0.099 \]

\[{t_{ok}}=\bar{t_{0k}}\pm \Delta{\bar{t_{0k}}}=17.72\pm 0.09 \, s\]

\subsection*{Рассчитаем $T_d = \bar{T_d} \pm \Delta \bar{T_d}$}

\[ \bar{T_d} = \frac{\bar{t_d}}{n} = 15.742/10 = 1,57 \, s \]
\[ \Delta \bar{T_d} = \frac{\Delta \bar{t_d}}{n} = 0.129/10 = 0.013 \]
\[ T_d = \bar{T_d} \pm \Delta \bar{T_d} = 1.570 \pm 0,013 \, s \]

\subsection*{Рассчитаем $T_k = \bar{T_k} \pm \Delta \bar{T_k}$}

\[ \bar{T_k} = \frac{\bar{t_k}}{n} = 21.9/10 = 2.2 \, s \]
\[ \Delta \bar{T_d} = \frac{\Delta \bar{t_d}}{n} = 0.207/10 = 0.02 \]
\[ T_k = \bar{T_k} \pm \Delta \bar{T_k} = 2.20 \pm 0.02 \, s \]

\subsection*{Рассчитаем время затухания}

\[ \tau = \frac{t_0}{\ln2} \]
\[ \bar{t_{d}} = \frac{\bar{t_{0d}}}{\ln2} = 19.526/0.693 = 28.18 \, s \]
\[ \bar{t_{k}} = \frac{\bar{t_{0k}}}{\ln2} = 17.722/0.693 = 25.57 \, s \]

\[ \tau_d = 28.18 \pm 1.4 \, s \]
\[ \tau_k = 25.57 \pm 0.14 \, s \]

\subsection*{Найдем собственную частоту колебаний \\ маятника без кольца и с кольцом}

\[ \bar{\omega_{0d}} = \frac{2\pi}{\bar{T_d}} = 4 \, s^{-1} \]
\[ \bar{\omega_{0k}} = \frac{2\pi}{\bar{T_k}} = 2.85 \, s^{-1} \]

\[ \Delta \bar{\omega_{0d}} = 0.033 \, s^{-1} \]
\[ \Delta \bar{\omega_{0k}} =  0.026 \, s^{-1}\]

\[ \omega_{0d} = 4.00 \pm 0.03 \, s^{-1} \]
\[ \omega_{0k} = 2.85 \pm 0.03 \, s^{-1} \]

\subsection*{Рассчитаем момент инерции кольца}

\[ I_k = \frac{m}{8}\biggl( D_{ex}^2 + D_{in}^2\biggr) = \frac{1.832}{8} \cdot (0.247^2 + 0.0585^2) = 0.015 \, kg\cdot m^2 \]

\subsection*{Рассчитаем момент инерции диска}

\[ I_d = \bar{I_d} \pm \Delta \bar{I_d} \]

\[ \bar{I_d} = \frac{I_k\bar{\omega_{0k}^2}}{\bar{\omega_{0d}^2}-\bar{\omega_{0k}^2}} = 0.015\]
Прологарифмировав данное выражение получим:
\[ \ln I_k + 2\ln \omega_{0k} = \ln I_d \cdot (\ln (\omega_{0d} + \omega_{0k} ) + \ln (\omega_{0d} - \omega_{0k})) \]

\[ \Delta \bar{I_d} = \sqrt{\biggl(\frac{dI_d}{d\omega_{0d}}\bigg \vert_{\bar{\omega_{0d}},\bar{\omega_{0k}}}\cdot \Delta \bar{\omega_{0d}}\biggr)^2 + \biggl(\frac{dI_d}{d\omega_{ok}}\bigg \vert_{\bar{\omega_{0d}},\bar{\omega_{0k}}}\cdot \Delta \bar{\omega_{0k}}}\biggr)^2 = 0.004\]
\[ \frac{dI_d}{d\omega_{0d}}\bigg \vert_{\bar{\omega_{0d}},\bar{\omega_{0k}}} = -1 \]
\[ \frac{dI_d}{d\omega_{0k}}\bigg \vert_{\bar{\omega_{0d}},\bar{\omega_{0k}}} = 1.42 \]

\[ I_d = \Delta I_d \pm \Delta \bar{I_d} = 0.015 \pm 0.004 \, kg\cdot m^2\]


\newpage

\end{document} 